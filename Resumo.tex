
	\section*{Resumo}
	
	Neste trabalho, realizamos medidas sistem�ticas dos efeitos Hall e Shubnikov-de Haas em fun��o do tempo de ilumina��o das amostras a fim de investigar as propriedades de transporte el�trico de um g�s bidimensional de el�trons confinado num po�o-qu�ntico de GaAs/InGaAs pr�ximo a pontos-qu�nticos de InAs introduzidos na barreira superior do po�o-qu�ntico (GaAs). N�s n�o observamos qualquer degrada��o expressiva da mobilidade eletr�nica devido a inser��o deles na heteroestrutura. Contudo, observamos diferentes varia��es das mobilidades qu�nticas de amostra para amostra, as quais atribuimos ao ac�mulo da tens�o mec�nica na camada de InAs. O comportamento das mobilidades qu�nticas e de transporte s�o discutidas no contexto da modula��o local dos perf�s das bandas de condu��o e de val�ncia pela camada de InAs.
	
	\section*{\foreign{Abstract}}
	{\slshape
	
	In this work, systematic Shubnikov-de Haas and Hall measurements as a function of the sample illumination time were used to investigate the transport properties of a two-dimensional electron gas (\textsc{2deg}) confined in GaAs/InGaAs quantum wells and close to InAs quantum-dots placed in the GaAs top barrier. We did not observe any expressive degradation of the electronic mobility due to the insertion of them in the heterostructure. However, we observed a different change of the quantum mobility of the occupied subbands from sample to sample, which was attributed to the accumulation of mechanical strain in the InAs layer. The behavior of the quantum and transport mobilities are discussed in the context of the local modulation of the band edges by the InAs layer.
	}